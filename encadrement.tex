

\section{Diffusion de la culture scientifique}

\subsection{IAPR Technical Committee ``Discrete Geometry''}
\label{sec:tc18}

Depuis fin    2003,   je  suis    co-r�sponsable  avec  \textsc{Annick
  Montanvert}   (Pr,  Universit� Pierre   Mend�s-France,  Grenoble) du
\emph{Technical                                             Committee}
18\footnote{\url{http://www.cb.uu.se/~tc18/}} de l'\emph{International
  Association                      on                          Pattern
  Recognition}\footnote{\url{http://www.iapr.org}} (IAPR).  Ce  comit�
international  a  pour objectif de  faciliter les  interactions et les
collaborations entre chercheurs  travaillant en g�om�trie discr�te  ou
dans  des   domaines   connexes (le   TC    regroupe actuellement   59
chercheurs).    Pour  cela,  un site   web  regroupant  des ressources
(documents, codes, jeux de donn�es, \ldots) a �t� cr��. De plus, le TC
participe � l'organisation de la conf�rence majeure en g�om�trie
discr�te : \emph{Discrete Geometry for Computer Imagery} (DGCI).

\subsection{Participation � des comit�s de programme et de lecture}

{\bf Participation � des comit�s de programme :}
\begin{itemize}
\item Discrete Geometry for Computer Imagery (DGCI) 2005\footnote{\url{http://www.sic.sp2mi.univ-poitiers.fr/dgci/}}
\item International Workshop on Combinatorial Image Analysis (IWCIA) 2004\footnote{\url{http://www.citr.auckland.ac.nz/~IWCIA04/}}
\item International Workshop on Combinatorial Image Analysis (IWCIA) 2006\footnote{\url{http://www.informatik.hu-berlin.de/IWCIA/}} 
\end{itemize}

{\bf Participation � des comit�s de lecture :}
\begin{itemize}
\item relecteur r�gulier des conf�rences DGCI, IWCIA et  International
  Conference on Pattern Recognition (ICPR)~;
\item relecteur pour les revues~:
  \begin{itemize}
  \item IEEE Trans. on Pattern Analysis and Machine Intelligence
  \item Discrete Applied Mathematics
  \item Pattern Recognition Letters
  \item Computer and Graphics   
  \end{itemize}
\end{itemize}


\subsection{Animation scientifique au sein de l'unit�}

Au sein de mon  unit�, j'ai particip� � l'organisation  des s�minaires
de l'axe 2  sur la p�riode 2003-2004.  Sur  les deux ann�es que couvre
ce  rapport  d'activit�, j'ai   eu l'occasion  de  pr�senter plusieurs
s�minaires   de  recherches  (Introduction �  la  g�om�trie  discr�te,
pr�sentation des travaux de l'ACI,  pr�sentation des articles SIGGRAPH
2005  au LIRIS) mais  aussi de formation (Introduction � \LaTeX/BibTex
avec \textsc{Eric Gu�rin}).


%%% Local Variables: 
%%% mode: latex
%%% TeX-master: "dossier_CV"
%%% TeX-master: "dossier_CV"
%%% End: 
