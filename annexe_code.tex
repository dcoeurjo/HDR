\chapter{R�alisations logicielles et diffusion}
\label{chap:real-logic-et}

Disponibles   sur le  site   du  Technical   Committee 18 de    l'IAPR
\url{http://www.cb.uu.se/~tc18/},   rubrique   \texttt{Code},   ou sur
\url{http://liris.cnrs.fr/david.coeurjolly}.

\vspace{1cm}


\begin{itemize}
\item \texttt{libvol, liblongvol, voltools} : Biblioth�que et outils
  de manipulation d'objets discrets.
\item \texttt{volgen} : outils  de  cr�ation d'objet discrets  simples
  (cube, sph�re,    cat�noid,\ldots). Cet   outils  fournit aussi  une
  signature  de l'objet   euclidien  sous-jacent  afin  d'�valuer  des
  estimateurs g�om�trique d'aire, de normale et de courbure.
\item \texttt{sedt} : impl�mentation de l'algorithme lin�aire en temps de transformation
  en distance euclidienne \cite{Hirata} (utilisant \texttt{libvol} et \texttt{liblongvol}).

\item  \texttt{DistanceTransform} :   projet contenant  les diff�rents
  codes pour le calcul de la transform�e en distance en temps lin�aire,
  la  transform�e en distance  inverse et l'extraction de l'axe m�dian
  aussi en temps lin�aires \citejournaux{dcoeurjo_pami_RDMA}
\item  \texttt{reconstruction} :  (coll.   avec  \aut{Loutfi Zerrarga}
  (M2R))   reconstruction  r�versible dans    des 4-courbes    simples
  \citejournaux{dcoeurjo_computergraphics}.
\item \texttt{Digital plane preimage analysis} : outils permettant de
  construire et de visualiser la pr�image d'un plan discret

\item Challenge \emph{Comparative Evaluation  of Digital Plane Segment
    Recognition Algorithms} (coordinateur) : infrastructure logicielle
  et   impl�mentations   d'algorithmes  de reconnaissance    de  plans
  discrets.   Ce challenge du TC18-IAPR a  pour  objectif de mettre en
  commun  les  diff�rents   d�veloppements  de plusieurs  laboratoires
  (actuellement LAIC Clermont-Ferrand et LIRIS Lyon).
\end{itemize}



%%% Local Variables: 
%%% mode: latex
%%% TeX-master: "hdr"
%%% End: 
