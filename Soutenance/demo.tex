\documentclass{beamer}
\usepackage[french]{babel}
\usepackage{beamerthemeliris}
\usepackage[latin1]{inputenc}
\usepackage{times}
\usepackage[T1]{fontenc}

\title[Guerre et Paix]
{Guerre et Paix}
%\subtitle{}

\author
[Leo N. Tolsto� - http://liris.cnrs.fr/leo.tolstoi]
{Leo N. Tolsto�%\inst{1}}
}

\institute%[XXX]
{
  %\inst{1}%
  {\bf Laboratoire d'InfoRmatique en Image et Syst�mes d'information} \\
  { \scriptsize{
  LIRIS UMR 5205 CNRS/INSA de Lyon/Universit� Claude Bernard Lyon 1/Universit� Lumi�re Lyon 2/Ecole Centrale de Lyon\\
  INSA de Lyon, b�timent J. Verne\\
  20, Avenue Albert Einstein - 69622 Villeurbanne cedex\\
  \url{http://liris.cnrs.fr}}
  }
}

\date%[KPT 2003]
{30 Novembre 2006}
%\subject{Informatik}

\AtBeginSection[]
{
  \begin{frame}<beamer>
    \frametitle{Plan}
    \tableofcontents[currentsection,currentsubsection]
  \end{frame}
}


\begin{document}

\begin{frame}[plain]
  \titlepage
\end{frame}

\begin{frame}
  \frametitle{Plan}
  \tableofcontents
\end{frame}

%***********************************************************************
% SECTION
\section{Introduction}
%***********************************************************************

% ----------------------------------------------------------------------
\begin{frame}
  	\frametitle{La lit�rature au LIRIS}
  	
  	\begin{itemize}
  	 \item La litt�rature russe
  	 	\begin{itemize}
			\item Fedor Dosto�evski
				\begin{itemize}
				\item Crime et Ch�timent
				\item Le joueur
				\item Les fr�res Karamazov
				\end{itemize}
			\item Leo N. Tolsto�
				\begin{itemize}
				\item Guerre et Paix
				\item Anna Kar�nine
				\end{itemize}
		\end{itemize}
	 \item La litt�rature Autrichienne
  	 	\begin{itemize}
			\item Elfriede Jelinek
			\item Robert Musil
				\begin{itemize}
				\item L'homme sans qualit�s
				\item Les D�sarrois de l'�l�ve T�rless			
				\end{itemize}
		\end{itemize}
  	\end{itemize}	
\end{frame}

%***********************************************************************
% SECTION
\section{La premi�re section}
%***********************************************************************

% ----------------------------------------------------------------------
\begin{frame}[label=structureetalert]
  	\frametitle{Structure et Alert}
  	
  	\begin{enumerate}
  	 \item The \structure{quick brown fox} jumps over the lazy dog.
  	 \item All work and no play makes jack a \alert{dull boy}.
  	 \item Lorem ipsum dolor sit amet, consectetuer adipiscing elit, sed diam nonummy nibh euismod tincidunt ut laoreet dolore magna aliquam erat volutpat. 
  	\end{enumerate}
  	
\end{frame}

% ----------------------------------------------------------------------
\begin{frame}
  	\frametitle{Structure et Alert}
  	
  	On peut aussi r�p�ter des frames.
  	
\end{frame}


% ----------------------------------------------------------------------
\againframe{structureetalert}


%***********************************************************************
% SECTION
\section{Une autre section}
%***********************************************************************

% ----------------------------------------------------------------------
\begin{frame}
  	\frametitle{Comment faire une animation}  	  	  	
  	
  	Ceci appari�t tout de suite. 
  	
  	\pause
  	Je crois que vous venez d'appuyer sur une touche.
  	
  	\pause
  	Et encore.
  	
\end{frame}

%***********************************************************************
% SECTION
\section{Resultats}
%***********************************************************************

% ----------------------------------------------------------------------
\begin{frame}
  	\frametitle{Comment faire un tableau}  	  	  	
  	
	\begin{center}
  	\begin{tabular}{l||c|c|c}
	M�thode &Class�e 1&Class�e 2&Class�e 3\\
	\hline
	\hline
	M�thode 1 & 18& 10& 36\\
	M�thode 2 & 13& 39& 12\\
	M�thode 3 & 33& 15& 16\\
	\hline
	Total & 64 & 64 & 64 \\
	\end{tabular} 
  	\end{center}
	
\end{frame}


% ----------------------------------------------------------------------
\begin{frame}
  	\frametitle{Comment faire une description}  	  	  	
	
	\begin{description}
	\item[$H_0$] La m�thode 3 est aussi efficace que les deux autres
	\item[$H_A$] La m�thode 3 est soit plus efficace soit moins efficace qu'une ou les deux autres m�thodes.
	\end{description}
		
	$$ 
		U \sim B(N,p) 
		\quad \quad \quad \quad N=64, \quad p=\frac{1}{3}
	$$
 
	$$
	P(U=33) \ = \ 
	\left (
	\begin{array}{c}
	N \\
	33 \\
	\end{array}
	\right )
	\pi^{p}(1-\pi)^{N-3}
	\ = \ 0.00111
	$$
	
\end{frame}

\end{document}


